\documentclass[letterpaper,10pt]{article}

\usepackage{graphicx}                                        
\usepackage{amssymb}                                         
\usepackage{amsmath}                                         
\usepackage{amsthm}                                          

\usepackage{alltt}                                           
\usepackage{float}
\usepackage{color}
\usepackage{url}
\usepackage{listings}

\usepackage{balance}


\usepackage{geometry}
\geometry{textheight=8.5in, textwidth=6in}

%random comment

\newcommand{\cred}[1]{{\color{red}#1}}
\newcommand{\cblue}[1]{{\color{blue}#1}}

%\newcommand{\toc}{\tableofcontents}

\title{CS 444 Assignment 1}
\author{Jared Wasinger, Soo-Min Yoo}


\parindent = 0.0 in
\parskip = 0.1 in

\begin{document}

\maketitle

\begin{abstract}
	This is an abstract
\end{abstract}
\newpage

\tableofcontents

\newpage

%input the pygmentized output of mt19937ar.c, using a (hopefully) unique name
%this file only exists at compile time. Feel free to change that.


\section{Log of commands for Qemu}


\textbf{open two terminal windows in terminal \#1:}\\
  cd /scratch/spring/2017/\\
  mkdir 14-05\\
  cd 14-05\\
  git clone git://git.yoctoproject.org/linux-yocto-3.14\\
  cd linux-yocto-3.14\\
  git checkout v3.14.26\\
  configure environment with this command: source /scratch/opt/environment-setup-i586-poky-linux.ocsh\\
  cp /scratch/spring2017/files/config-3.14.26-yocto-qemu .config\\
  make menuconfig\\
  press / and type in LOCALVERSION, press enter\\
  hit 1, press enter, edit value to -14-05-hw1\\
  make -j4 all\\
  cd ../\\
  run gdb\\

\textbf{in terminal \#2:}\\
  cd /scratch/spring/2017/14-05\\
  cp /scratch/spring2017/files/bzImage-qemux86.bin . and cp /scratch/spring2017/files/core-image-lsb-sdk-qemux86.ext3 .\\
  run qemu with: qemu-system-i386 -gdb tcp::5645 -S -nographic -kernel bzImage-qemux86.bin -drive file=core-image-lsb-sdk-qemux86.ext3,if=virtio -enable-kvm -net none -usb -localtime --no-reboot --append "root=/dev/vda rw console=ttyS0 debug"\\

\textbf{back in terminal \#1 in gdb:}\\
  target remote :5645\\
  continue\\
  you'll be asked for qemu login: type in root, enter\\
  type uname -a to see kernel name\\
  reboot\\

\textbf{in terminal \#2:}
  run qemu again with the following (note that it's different from the one before!):\\
    qemu-system-i386 -gdb tcp::5645 -S -nographic -kernel linux-yocto-3.14/arch/x86/boot/bzImage -drive file=core-image-lsb-sdk-qemux86.ext3,if=virtio -enable-kvm -net none -usb -localtime --no-reboot --append "root=/dev/vda rw console=ttyS0 debug"\\
  
\textbf{in terminal \#1 in gdb:}\\
  target remote :5645\\
  continue\\
  qemu login: type in root, enter\\
  type uname -a to see kernel name, which now has -14-05-hw1 appended to the end\\
  reboot\\

\section{Explanation of Qemu Flags}

\begin{itemize}
\item{\textbf{-gdb tcp::???? }}\\
open a gdbserver on TCP port ????\\

\item{\textbf{-S }}\\
Do not start CPU at startup (you must type ’c’ in the monitor).\\

\item{\textbf{-nographic }}\\
Normally, QEMU uses SDL to display the VGA output. With this option, you can totally disable graphical output so that QEMU is a simple command line application. The emulated serial port is redirected on the console and muxed with the monitor (unless redirected elsewhere explicitly). Therefore, you can still use QEMU to debug a Linux kernel with a serial console. Use C-a h for help on switching between the console and monitor.\\

\item{\textbf{-kernel bzImage}}\\
Use bzImage as kernel image. The kernel can be either a Linux kernel or in multiboot format.\\

\item{\textbf{-drive file=core-image-lsb-sdk-qemux86.ext3,if=virtio }}\\
specify drive file to open (?)\\

\item{\textbf{-enable-kvm }}\\
Enable KVM full virtualization support. This option is only available if KVM support is enabled when compiling.\\

\item{\textbf{-net none}}\\
Indicate that no network devices should be configured. It is used to override the default configuration (-net nic -net user) which is activated if no -net options are provided.\\

\item{\textbf{-usb}}\\
Enable the USB driver\\

\item{\textbf{-localtime}}\\
Set the real time clock to local time (?)\\

\item{\textbf{--no-reboot}}\\
Exit instead of rebooting.

\item{\textbf{--append cmdLine}}\\
Use cmdLine as kernel command line\\

\end{itemize}
\section{Concurrency Problem Questions}

\begin{enumerate}
  \item{\textbf{What do you think the main point of this assignment is?}}\\
  The purpose of this assignment was for us to familiarize ourselves with concurrent programming techniques using a problem that mimics real-world concurrent systems.\\
  \item{\textbf{How did you personally approach the problem? Design decisions, algorithm, etc.}}\\
  The description of the behavior from the assignment specification was enough information for me to create a simple multi-threaded buffer.
  \item{\textbf{How did you ensure your solution was correct? Testing details, for instance.}}
  Log output from the executable gives accurate information about the operations that happen (consume/produce) and the size of the buffer after each operation.  By reading log output and ensuring the output indicates that the operations being performed correctly.\\
  \item{\textbf{What did you learn?}}\\
  This assignment was great for re-familiarizing myself with POSIX compliant C multithreading mechanisms.\\
\end{enumerate}
\section{Version Control Log}


\section{Work Log}
\begin{itemize}
  \item{Designing and building solution for producer/consumer problem in C - Completed by Jared Wasinger 4/16}
  \item{Building the VM, documenting the commands - Completed by Soo-min Yoo 4/19}
  \item{LaTex Writeup and Assignment submission - Completed by Jared Wasinger 4/21}
\end{itemize}

\section{Code Listing}
\subsection{producer\_consumer.h}
\lstset{language=C,
                basicstyle=\ttfamily,
                keywordstyle=\color{blue}\ttfamily,
                stringstyle=\color{red}\ttfamily,
                commentstyle=\color{green}\ttfamily,
                morecomment=[l][\color{magenta}]{\#}
}

\begin{lstlisting}
#ifndef PRODUCER_CONSUMER_H
#define PRODUCER_CONSUMER_H

#define BUF_MAX_LEN 30

#include<stdio.h>
#include<unistd.h>
#include<stdlib.h>
#include<pthread.h>
#include<errno.h>
#include<stdint.h>
#include<x86intrin.h>

#include "genrand.h"

typedef struct {
  int Id;
  int WaitTime;
} Item;

/*
 * SyncBuffer is a structure which implements the functionality of a threadsafe stack in the context of the producer consumer problem:
 *    X producers, Y consumers where X,Y > 0
 */
typedef struct {
  Item *buffer;
  int buffer_len;
  pthread_mutex_t mutex;
} SyncBuffer;

void produce_item();

void consume_item();

int init_sync_buffer(SyncBuffer** rw_buf);

int32_t gen_rdrand();

Item gen_item();

void capture_error(int ret);

#endif

\end{lstlisting}
\newpage

\subsection{producer\_consumer.c}

\begin{lstlisting}
#include "producer_consumer.h"

int next_id = 0;

void fatal_error(char *error_str) {
	printf(error_str);
	exit(-1);
}

/*
 * Wrapper around inline x86 asm to check hardware support for RDRAND
 */
int get_rdrand_support() {
	unsigned int eax;
	unsigned int ebx;
	unsigned int ecx;
	unsigned int edx;

	char vendor[13];
	
	eax = 0x01;

	__asm__ __volatile__(
						 "cpuid;"
						 : "=a"(eax), "=b"(ebx), "=c"(ecx), "=d"(edx)
						 : "a"(eax)
						 );
	
	if(ecx & 0x40000000){
		return 1;
	}
	else{
		return 0;
	}
}

/*
 * TODO: make threadsafe
 *
 * Fail if provided return value is not 0 and print a message about the error.
 *
 */
void capture_error(int ret) {
	if(ret != 0) {
		int errno_saved = errno;
		fprintf(stderr, "An error occurred!");
		fprintf(stderr, "The error value is %d\n", errno_saved);
		exit(-1);
	}
}

void fail_if_error(int ret) {
	if (ret != 0) {
		exit(-1);
	}
}

/*
 * SyncBuffer Init function
 */
int init_sync_buffer(SyncBuffer** rw_buf) {
	int ret = 0;

	SyncBuffer *new_buf = malloc(sizeof(SyncBuffer));
	if (new_buf == NULL) {
		fatal_error("...");
	}

	new_buf->buffer = malloc(sizeof(BUF_MAX_LEN));
	if (new_buf->buffer == NULL) {
		fatal_error("...");
	}

	new_buf->buffer_len = 0;

	ret = pthread_mutex_init(&(new_buf->mutex), NULL);
	if(ret != 0) {
		fatal_error("...");
	}

	*rw_buf = new_buf;
	return 0;
}

int pop_buffer(SyncBuffer* buf) {
	int ret = 0;

	if (buf->buffer_len == 0) {
		return 0;
	}

	ret = pthread_mutex_lock(&(buf->mutex));

	//buffer has been modified by another thread while waiting for mutex to unlock
	if (buf->buffer_len == 0) {
		return 0;
	}

	
	buf->buffer_len--;

	capture_error(pthread_mutex_unlock(&(buf->mutex)));

	return 1;
}

void push_buffer(SyncBuffer* buf, Item *item) {
	if (buf->buffer_len == BUF_MAX_LEN) {
		return;
	}

	while(1) {
		if (buf->buffer_len == BUF_MAX_LEN) {
			continue;
		}

		capture_error(pthread_mutex_lock(&(buf->mutex)));
		
		buf->buffer[buf->buffer_len] = *item;
		buf->buffer_len++;

		capture_error(pthread_mutex_unlock(&(buf->mutex)));
		break;
	}
}

/*
 * Generate an item and push it onto the buffer.  If the buffer is full, block until an item
 * is removed.  Then, add the new item
 */
void produce_item(SyncBuffer *buf) {
	while(1) {
		if (buf->buffer_len != BUF_MAX_LEN) {
			//wait 3-7 seconds before producing an item
			int32_t rand_sleep_value = (gen_rdrand() % 4)+3;
			sleep(rand_sleep_value);

			Item item = gen_item();
			push_buffer(buf, &item);

			printf("item produced\n");
		}
	}
}

/*
 * Consume the next item.  If the stack is empty, block until another item is added
 * and then remove it
 */
void consume_item(SyncBuffer *buf) {
	int ret = -1;
	while(1) {
		if (buf->buffer_len != 0) {
			//remove an item from the buffer
			ret = pop_buffer(buf);
			if(ret == 1) {
				printf("item consumed\n");
			}

			//decrement buffer length
			pthread_mutex_unlock(&(buf->mutex));
			break;
		}
	}
}

Item gen_item() {
	Item item;
	item.Id = next_id;
	next_id++;
	item.WaitTime = gen_rdrand();
}

/*
 * generate a random 32bit integer using hardware-based rdrand
 *
 */
int32_t gen_rdrand() {
	if(get_rdrand_support() != 0) {
		int result;
		_rdrand32_step(&result);
		return result;
	} else {
		return (int32_t)genrand_int32();
	}
}
\end{lstlisting}
\newpage

\subsection{genrand.h}
\begin{lstlisting}
#ifndef GENRAND_H
#define GENRAND_H

#include <stdio.h>

/* Period parameters */
#define N 624
#define M 397
#define MATRIX_A 0x9908b0dfUL   /* constant vector a */
#define UPPER_MASK 0x80000000UL /* most significant w-r bits */
#define LOWER_MASK 0x7fffffffUL /* least significant r bits */

unsigned long genrand_int32(void);

void init_genrand(unsigned long s);

#endif

\end{lstlisting}
\newpage

\subsection{genrand.c}
\begin{lstlisting}
#include "genrand.h"

static unsigned long mt[N]; /* the array for the state vector  */
static int mti=N+1; /* mti==N+1 means mt[N] is not initialized */

/* generates a random number on [0,0xffffffff]-interval */
unsigned long genrand_int32(void)
{
  unsigned long y;
  static unsigned long mag01[2]={0x0UL, MATRIX_A};
  /* mag01[x] = x * MATRIX_A  for x=0,1 */

  if (mti >= N) { /* generate N words at one time */
    int kk;

    if (mti == N+1)   /* if init_genrand() has not been called, */
      init_genrand(5489UL); /* a default initial seed is used */

    for (kk=0;kk<N-M;kk++) {
      y = (mt[kk]&UPPER_MASK)|(mt[kk+1]&LOWER_MASK);
      mt[kk] = mt[kk+M] ^ (y >> 1) ^ mag01[y & 0x1UL];
    }
    for (;kk<N-1;kk++) {
      y = (mt[kk]&UPPER_MASK)|(mt[kk+1]&LOWER_MASK);
      mt[kk] = mt[kk+(M-N)] ^ (y >> 1) ^ mag01[y & 0x1UL];
    }
    y = (mt[N-1]&UPPER_MASK)|(mt[0]&LOWER_MASK);
    mt[N-1] = mt[M-1] ^ (y >> 1) ^ mag01[y & 0x1UL];

    mti = 0;
  }

  y = mt[mti++];

  /* Tempering */
  y ^= (y >> 11);
  y ^= (y << 7) & 0x9d2c5680UL;
  y ^= (y << 15) & 0xefc60000UL;
  y ^= (y >> 18);

  return y;
}

/* initializes mt[N] with a seed */
void init_genrand(unsigned long s)
{
  mt[0]= s & 0xffffffffUL;
  for (mti=1; mti<N; mti++) {
    mt[mti] =
      (1812433253UL * (mt[mti-1] ^ (mt[mti-1] >> 30)) + mti);
    /* See Knuth TAOCP Vol2. 3rd Ed. P.106 for multiplier. */
    /* In the previous versions, MSBs of the seed affect   */
    /* only MSBs of the array mt[].                        */
    /* 2002/01/09 modified by Makoto Matsumoto             */
    mt[mti] &= 0xffffffffUL;
    /* for >32 bit machines */
  }
}

\end{lstlisting}
\newpage

\subsection{main.c}
\begin{lstlisting}
#include<stdio.h>
#include<unistd.h>
#include<stdlib.h>
#include<pthread.h>
#include<getopt.h>

#include "producer_consumer.h"

void *consumer_main(void *sync_buf) {
  while(1) {
    consume_item((SyncBuffer *)sync_buf);
  }
}

void *producer_main(void *sync_buf) {
  while(1) {
    produce_item((SyncBuffer *)sync_buf);
  }
}

int main(int argc, char **argv) {
  SyncBuffer *sync_buf = NULL;
  int num_consumers, num_producers, arg;
  num_consumers = num_producers = arg = -1;
  pthread_t *consumers = NULL;
  pthread_t *producers = NULL;

  while((arg = getopt(argc, argv, "c:p:")) != -1) {
    switch(arg) {
      case 'c':
        num_producers = atoi(optarg);
        break;
      case 'p':
        num_consumers = atoi(optarg);
        break;
    }
  }

  if(num_consumers == -1 || num_producers == -1 || num_producers <= 0 || num_consumers <= 0) {
    printf("Argument error...\n");
    exit(-1);
  }

  capture_error(init_sync_buffer(&sync_buf));

  consumers = malloc(sizeof(pthread_t) * num_consumers);
  producers = malloc(sizeof(pthread_t) * num_producers);

  for(int i = 0; i < num_producers; i++) {
    capture_error(pthread_create(producers + i,
            NULL,
            producer_main,
            sync_buf));
  }
  
  for(int i = 0; i < num_producers; i++) {
    capture_error(pthread_create(consumers + i,
            NULL,
            consumer_main,
            sync_buf));
  }

  for(int i = 0; i < num_producers; i++) {
    pthread_join(producers[i], NULL);
  }

  for(int i = 0; i < num_consumers; i++) {
    pthread_join(consumers[i], NULL);
  }

  return 0;
}
\end{lstlisting}
\newpage

\end{document}
